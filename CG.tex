\documentclass{article}
%
\usepackage{amsmath}
\usepackage{amsthm}
\usepackage{hyperref}
%
\DeclareMathOperator{\Tr}{Tr}
\newtheorem{theorem}{Theorem}
%
\title{An Interest Observation about Clifford Group}
\author{Guangliang He
  \thanks{\href{mailto:guangliang.he@gmail.com}
    {Email: guangliang.he@gmail.com}}}
\date{\today}

\begin{document}
\maketitle

\section{Definitions}
The Pauli group $\mathcal P_n$ on $n$ qubits is defined as
\begin{equation}
  \mathcal P_n = \{e^{i\theta\pi/2}\sigma^{(1)}\otimes\cdots\otimes\sigma^{(n)}
  \mid \theta\in\{0,1,2,3\}, \sigma^{(k)} \in \{I, X, Y, Z\}\},
\end{equation}
where
\begin{alignat}{3}
  I &= \begin{bmatrix} 1 & 0 \\ 0 & 1\end{bmatrix}, \\
  X &= \begin{bmatrix} 0 & 1 \\ 1 & 0\end{bmatrix}, \\
  X &= \begin{bmatrix} 0 & -i \\ i & 0\end{bmatrix}, \\
  Z &= \begin{bmatrix} 1 & 0 \\ 0 & 1\end{bmatrix}.
\end{alignat}
The group 
$\mathcal N(\mathcal P_n) = \{U\in U(2^n)\mid U\mathcal P_nU^\dagger
= \mathcal P_n\}$ is called the normalizer of $\mathcal P_n$.
Ignoring the global phase, we define the Clifford group as the quotient
group $\mathcal C_n = \mathcal N(\mathcal P_n)/U(1)$.
It is known that\cite{Koenig_2014}
$\mid\mathcal C_n\mid = 2^{n^2+2n}\prod_{j=1}^n(4^j-1)$.

\section{Random sampling Clifford group and density matrix}
Random sampling operator from the Clifford group is often used in quantum
information.  We know that the result of apply an operator $U$ to an
initial state represented by the density matrix $\rho$, is
$U\rho U^\dagger$.  If we randomly draw an operator from the uniformly
distributed set $\mathcal C_n$, the resulting state is
\begin{equation}
  \label{eq:rho_prime}
  \rho' = \frac{1}{\mid\mathcal C_n\mid}\sum_{U\in\mathcal C_n}U\rho U^\dagger.
\end{equation}

\begin{theorem}
  Let $\rho$ be the density matrix of a quantum state of $n$ qubits.
  After applying a randomly drawn unitary operator from the Clifford group
  $\mathcal C_n$, the density matrix of the resulting state, as shown in
  Eq. (\ref{eq:rho_prime}), is
  \begin{equation}
    \rho' = \frac{I_{2^n}}{2^n}.
  \end{equation}
  In other words, regardless the initial state, the resulting state is a
  maximally mixed state.
\end{theorem}
\begin{proof}
  Let $\Sigma = \{I, X, Y, Z\}$,
  for any $\sigma^{(1)}, \ldots, \sigma^{(n)}\in\Sigma$, it is easy to
  see that  $\sigma^{(1)}\otimes\cdots\otimes\sigma^{(n)}\in \mathcal C_n$.
  Thus we have
  \begin{equation}
    \sum_{U\in\mathcal C_n} U\rho U^\dagger =
    \frac{1}{4^n}\sum_{\sigma^{(1)}\in\Sigma}\cdots\sum_{\sigma^{(n)}\in\Sigma}\sum_{U\in\mathcal C_n}
    U\sigma^{(1)}\otimes\cdots\sigma^{(n)}\rho\sigma^{(1)}\otimes\cdots\sigma^{(n)}U^\dagger.
  \end{equation}
  Generalize Eq.~(8.101) of 
  Nielsen and Chuang\cite{Nielsen_Chuang_2010} into $n$ qubit case.
  That is, for any $2^n\times2^n$ matrix $M$,
  \begin{equation}
    \frac{1}{4^n}\sum_{\sigma^{(1)}\in\Sigma}\cdots\sum_{\sigma^{(n)}\in\Sigma}
    \sigma^{(1)}\otimes\cdots\otimes\sigma^{(n)}M
    \sum_{\sigma^{(1)}\in\Sigma}\cdots\sum_{\sigma^{(n)}\in\Sigma}
    \sigma^{(1)}\otimes\cdots\otimes\sigma^{(n)} = \frac{\Tr M}{2^n}I_{2^n}.
    \label{eq:sum_Sigma}
  \end{equation}
  Thus
  \begin{equation}
    \sum_{U\in\mathcal C_n}U\rho U^\dagger
    = \sum_{U\in\mathcal C_n}\frac{UI_{2^n}U^\dagger}{2^n}
    = \frac{\mid\mathcal C_n\mid I_{2^n}}{2^n},
  \end{equation}
  Here we used the fact $\Tr\rho = 1$.  Substitute into Eq.~(\ref{eq:rho_prime}),
  we have the final result
  \begin{equation}
    \rho' = \frac{I_{2^n}}{2^n}.
  \end{equation}
\end{proof}

\appendix
\section{Proof of Eq.~(\ref{eq:sum_Sigma})}
We will prove Eq.~(\ref{eq:sum_Sigma}) by induction.
For the case $n=1$, it is equivalent to Eq.~(8.101) of Nielsen and Chuang,
thus it is correct.
Assume Eq.~(\ref{eq:sum_Sigma}) is correct for $n = k$, we will prove it
is correct for $n = k+1$. Let's first consider the case matrix $M$ is
a tensor product $M = M^{(K)}\otimes M^{(k+1)}$ where $M^{(K)}$ and $M^{(k+1)}$
are $2^k\times2^k$ and $2\times2$ matrices, respectively. Utilize
Eq.~(\ref{eq:sum_Sigma}) for $n = k$ for summation of $\sigma^{(1)}$
to $\sigma^{(k)}$, and again utlize Eq.~(\ref{eq:sum_Sigma}) for $n = 1$
for summation of $\sigma^{(k+1)}$, we have
\begin{align}
  & \frac{1}{4^{k+1}}\sum_{\sigma^{(1)}\in\Sigma}\cdots\sum_{\sigma^{(k+1)}\in\Sigma}
  \sigma^{(1)}\otimes\cdots\otimes\sigma^{(k+1)} \nonumber\\
  & \times M\times \sum_{\sigma^{(1)}\in\Sigma}\cdots\sum_{\sigma^{(k+1)}\in\Sigma}
  \sigma^{(1)}\otimes\cdots\otimes\sigma^{(k+1)} \nonumber\\
  =& \frac{\Tr M^{(K)}I_{2^k}}{2^k}\otimes\frac{\Tr M^{(k+1)}I_2}{2}
  \nonumber\\
  =&\frac{\Tr M}{2^{k+1}}I_{2^{k+1}}.
\end{align}
The Eq.~(\ref{eq:sum_Sigma}) holds for $M = M^{(K)}\otimes M^{(k+1)}$.
Since any $2^{k+1}\times2^{k+1}$ matrix can always be decomposed to a
sum of such tensor products, thus, it is true for any matrix $M$.
We conclude the proof for $n = k+1$.  By the principle of induction,
Eq.~(\ref{eq:sum_Sigma}) holds for any integer $n \ge 1$.

%%%%%%%%%%%%%%%%%%%%%%%%%%%%%%
\bibliographystyle{plain}
\bibliography{../BIBTEX/mybib}
%%%%%%%%%%%%%%%%%%%%%%%%%%%%%%
\end{document}
