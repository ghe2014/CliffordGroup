\documentclass{article}
%
\usepackage{amsmath}
\usepackage{amsthm}
\usepackage{hyperref}
%
\newtheorem{theorem}{Theorem}
%
\title{An Interest Observation about Clifford Group}
\author{Guangliang He
  \thanks{\href{mailto:guangliang.he@gmail.com}
    {Email: guangliang.he@gmail.com}}}
\date{\today}

\begin{document}
\maketitle

\section{Definitions}
The Pauli group $\mathcal P_n$ on $n$ qubits is defined as
\begin{equation}
  \mathcal P_n = \{e^{i\theta\pi/2}\sigma^{(1)}\otimes\cdots\otimes\sigma^{(n)}
  \mid \theta\in\{0,1,2,3\}, \sigma^{(k)} \in \{I, X, Y, Z\}\},
\end{equation}
where
\begin{alignat}{3}
  I &= \begin{bmatrix} 1 & 0 \\ 0 & 1\end{bmatrix}, \\
  X &= \begin{bmatrix} 0 & 1 \\ 1 & 0\end{bmatrix}, \\
  X &= \begin{bmatrix} 0 & -i \\ i & 0\end{bmatrix}, \\
  Z &= \begin{bmatrix} 1 & 0 \\ 0 & 1\end{bmatrix}.
\end{alignat}
The group 
$\mathcal N(\mathcal P_n) = \{U\in U(2^n)\mid U\mathcal P_nU^\dagger
= \mathcal P_n\}$ is called the normalizer of $\mathcal P_n$.
Ignoring the global phase, we define the Clifford group as the quotient
group $\mathcal C_n = \mathcal N(\mathcal P_n)/U(1)$.
It is known that\cite{Koenig_2014}
$\mid\mathcal C_n\mid = 2^{n^2+2n}\prod_{j=1}^n(4^j-1)$.

\section{Random sampling Clifford group and density matrix}
Random sampling operator from the Clifford group is often used in quantum
information.  If we apply such an operator $U$ to an initial state
represented by the density matrix $\rho$, the resulting state is
\begin{equation}
  \label{eq:rho_prime}
  \rho' = \frac{1}{\mid\mathcal C_n\mid}\sum_{U\in\mathcal C_n}U\rho U^\dagger.
\end{equation}

\begin{theorem}
  Let $\rho$ be the density matrix of a quantum state of $n$ qubits.
  After applying a randomly drawn unitary operator from the Clifford group
  $\mathcal C_n$, the density matrix of the resulting state, as shown in
  Eq. (\ref{eq:rho_prime}), is
  \begin{equation}
    \rho' = \frac{I_{2^n}}{2^n}.
  \end{equation}
  In other words, regardless the initial state, the resulting state is a
  maximally mixed state.
\end{theorem}
\begin{proof}
  First of all, because $X^{\otimes n}\in\mathcal C_n$, we have
  \begin{equation}
    \sum_{U\in\mathcal C_n} U\rho U^\dagger =
    \sum_{U\in\mathcal C_n} UX^{\otimes n}\rho X^{\otimes n}U^\dagger.
  \end{equation}
  Similarly for $Y^{\otimes n}$ and $Z^{\otimes n}$. This leads to
  \begin{eqnarray}
    &   & \sum_{U\in\mathcal C_n} U\rho U^\dagger \nonumber \\
    & = & \frac14\sum_{U\in\mathcal C_n}\left(U\rho U^\dagger +
    UX^{\otimes n}\rho X^{\otimes n}U^\dagger +
    UY^{\otimes n}\rho Y^{\otimes n}U^\dagger +
    UZ^{\otimes n}\rho Z^{\otimes n}U^\dagger\right)
    \nonumber \\
    & = & \frac14\sum_{U\in\mathcal C_n}U\left(
    \rho + X^{\otimes n}\rho X^{\otimes n} + Y^{\otimes n}\rho Y^{\otimes n}
    Z^{\otimes n}\rho Z^{\otimes n} \right)U^\dagger.
  \end{eqnarray}
  It is straight forward to generalize Eq.~(8.101) of 
  Nielsen and Chuang\cite{Nielsen_Chuang_2010} into $n$ qubit case.
  That is, for any $n$ qubit density matrix $\rho$,
  \begin{equation}
    \frac14\left(\rho + X^{\otimes n}\rho X^{\otimes n}
    + Y^{\otimes n}\rho Y^{\otimes n} + Z^{\otimes n}\rho Z^{\otimes n}\right)
    = \frac{I_{2^n}}{2^n}.
  \end{equation}
  Thus
  \begin{equation}
    \sum_{U\in\mathcal C_n}U\rho U^\dagger
    = \sum_{U\in\mathcal C_n}\frac{UI_{2^n}U^\dagger}{2^n}
    = \frac{\mid\mathcal C_n\mid I_{2^n}}{2^n}.
  \end{equation}
  Substitute into Eq.~(\ref{eq:rho_prime}), we have the final result
  \begin{equation}
    \rho' = \frac{I_{2^n}}{2^n}.
  \end{equation}
\end{proof}


%%%%%%%%%%%%%%%%%%%%%%%%%%%%%%
\bibliographystyle{plain}
\bibliography{../BIBTEX/mybib}
%%%%%%%%%%%%%%%%%%%%%%%%%%%%%%
\end{document}
